\documentclass[]{article}
\usepackage{lmodern}
\usepackage{amssymb,amsmath}
\usepackage{ifxetex,ifluatex}
\usepackage{fixltx2e} % provides \textsubscript
\ifnum 0\ifxetex 1\fi\ifluatex 1\fi=0 % if pdftex
  \usepackage[T1]{fontenc}
  \usepackage[utf8]{inputenc}
\else % if luatex or xelatex
  \ifxetex
    \usepackage{mathspec}
  \else
    \usepackage{fontspec}
  \fi
  \defaultfontfeatures{Ligatures=TeX,Scale=MatchLowercase}
\fi
% use upquote if available, for straight quotes in verbatim environments
\IfFileExists{upquote.sty}{\usepackage{upquote}}{}
% use microtype if available
\IfFileExists{microtype.sty}{%
\usepackage{microtype}
\UseMicrotypeSet[protrusion]{basicmath} % disable protrusion for tt fonts
}{}
\usepackage[margin=1in]{geometry}
\usepackage{hyperref}
\hypersetup{unicode=true,
            pdftitle={README},
            pdfborder={0 0 0},
            breaklinks=true}
\urlstyle{same}  % don't use monospace font for urls
\usepackage{graphicx,grffile}
\makeatletter
\def\maxwidth{\ifdim\Gin@nat@width>\linewidth\linewidth\else\Gin@nat@width\fi}
\def\maxheight{\ifdim\Gin@nat@height>\textheight\textheight\else\Gin@nat@height\fi}
\makeatother
% Scale images if necessary, so that they will not overflow the page
% margins by default, and it is still possible to overwrite the defaults
% using explicit options in \includegraphics[width, height, ...]{}
\setkeys{Gin}{width=\maxwidth,height=\maxheight,keepaspectratio}
\IfFileExists{parskip.sty}{%
\usepackage{parskip}
}{% else
\setlength{\parindent}{0pt}
\setlength{\parskip}{6pt plus 2pt minus 1pt}
}
\setlength{\emergencystretch}{3em}  % prevent overfull lines
\providecommand{\tightlist}{%
  \setlength{\itemsep}{0pt}\setlength{\parskip}{0pt}}
\setcounter{secnumdepth}{0}
% Redefines (sub)paragraphs to behave more like sections
\ifx\paragraph\undefined\else
\let\oldparagraph\paragraph
\renewcommand{\paragraph}[1]{\oldparagraph{#1}\mbox{}}
\fi
\ifx\subparagraph\undefined\else
\let\oldsubparagraph\subparagraph
\renewcommand{\subparagraph}[1]{\oldsubparagraph{#1}\mbox{}}
\fi

%%% Use protect on footnotes to avoid problems with footnotes in titles
\let\rmarkdownfootnote\footnote%
\def\footnote{\protect\rmarkdownfootnote}

%%% Change title format to be more compact
\usepackage{titling}

% Create subtitle command for use in maketitle
\newcommand{\subtitle}[1]{
  \posttitle{
    \begin{center}\large#1\end{center}
    }
}

\setlength{\droptitle}{-2em}

  \title{README}
    \pretitle{\vspace{\droptitle}\centering\huge}
  \posttitle{\par}
    \author{}
    \preauthor{}\postauthor{}
    \date{}
    \predate{}\postdate{}
  

\begin{document}
\maketitle

\section{Process for Creating the Annual Estimate of Visitor Use of
Regional
Parks}\label{process-for-creating-the-annual-estimate-of-visitor-use-of-regional-parks}

\subsection{Step 0: Download and import data; tidy and write out to send
to
agencies}\label{step-0-download-and-import-data-tidy-and-write-out-to-send-to-agencies}

Data is entered in surveys on the Met Council's Qualtrics account. Login
info for the Qualtrics account is available at
N:/CommDev/Research/Research/Parks\_Regional. There are 10 separate
regional parks agencies, each of which enter their data in a separate
form, resulting in 10 separate datasets in Qualtrics. Each of these
should be downloaded as csv file and all should be stored in one folder
together to permit simultaneous import. In this step, these dataframes
are also slightly tidied (a Qualtrics log-related non-observation row
and several Qualtrics log-related variables are removed) and written out
to a folder entitled `Sets to Send for Accuracy Audit'. These datasets
will be sent to agencies with a request that they verify the data
entered against the data counts on their hard copies.

\subsection{Step 1: Bind all agencies' data
together}\label{step-1-bind-all-agencies-data-together}

Each of the 10 forms provided for the agencies have different numbering
conventions, and the datasets are imported with the question number as
the variable name, or heading. In this step, variables are renamed with
uniform, intuitive names, and bound together so that all steps following
are performed across all data at once.

\subsection{Step 2: Create a day type variable (weekend/holiday or
weekday); create a variable with the total number of days for that day
type in the
season}\label{step-2-create-a-day-type-variable-weekendholiday-or-weekday-create-a-variable-with-the-total-number-of-days-for-that-day-type-in-the-season}

The day type is an intermediary step between the recorded date in the
raw data and coding the total number of days in the season that year
that are that day type, as well as assigning some entrance usage
classifications. The day type variable itself is not used in the
analysis - only the total number of days in the season that are the same
day type as that date (weekend/holiday or weekday). For example, most
summer seasons have 30-some weekend/holiday days, and 60-some weekdays.
2017 had 33 weekend/holiday days (including the Saturday and Sunday
preceding Memorial Day), and 67 weekdays. Because usage of parks varies
dramatically between work and non-work days, the number of visitors
counted at a particular entrance of a trail/park is multiplied only
across the number of days in the season that fit in the same category.
For example, if 17 visitors visited the Northeast Diagonal Regional
Trail on a non-holiday Tuesday at 10am in 2017, then 17 would get
multiplied by 67 (the number of weekdays in the 2017 season) to scale
from usage at the trail/entrance/day/time level to trail/entrance/time
usage at the seasonal level. The multiplication is performed in step 7,
but is explained here for clarity's sake.

\subsection{Step 2a: Add numeric Park Identifier to dataset (used in
step 3 to add park entrance usage
classifications)}\label{step-2a-add-numeric-park-identifier-to-dataset-used-in-step-3-to-add-park-entrance-usage-classifications}

Why convert from park name to a park ID? Mainly, this helps one match
the park's use estimate for the current year to that park's use
estimates for past years (in step 7). Some parks or trails change names
but geographically remain the same; therefore, analyzing park/trail data
using a unique ID for each park/trail rather than their unique title
makes joining to estimates for past years' usages simpler.

\subsection{Step 3: Add Park Entrance Usage Classifications (High,
Medium or
Low)}\label{step-3-add-park-entrance-usage-classifications-high-medium-or-low}

Park entrance usage classifications are determined by park agencies, and
are intra-park comparable; that is, what is a `high' use entrance on one
park or trail may be a `low' use entrance on another park or trail (the
Visitor Use Study, conducted every 5 years, then determines the
multipliers for the usage classes by park by sampling from an entrance
of each usage class at each park). These classifications are, like the
day type variable in the previous step, an intermediary step between the
recorded park/entrance in the raw data and the expansion factor assigned
in step 6 and used for multiplication in step 7. The expansion factors
determine how to scale counts at the trail/day/time level to usage at
the park level on that day type for all trails of that usage
classification.

\subsection{Step 4: Add persons per vehicle and persons per bus
multipliers}\label{step-4-add-persons-per-vehicle-and-persons-per-bus-multipliers}

The persons per vehicle and persons per bus multipliers are defined by
the 5-Year Visitor Use Study; the vehicle and bus multipliers used here
are defined by the 2016 Visitor Study and 1998-1999 Visitor Study,
respectively, as there wasn't a representative sample of bus travel to
parks in either the 2008 or 2016 Visitor Study. The multipliers are used
to determine number of individuals traveling by car or bus into a park
(annual samples count only vehicles or buses, and not the individuals
within these transportation modes, according to sampling instructions).
These multipliers are defined at the park level, and are used to scale
up bus and vehicle counts only (which occurs in step 5).

\subsection{Step 5: Calculate total visitors in sample at each park
entrance by date and time of day \& assign expansion
factors}\label{step-5-calculate-total-visitors-in-sample-at-each-park-entrance-by-date-and-time-of-day-assign-expansion-factors}

In this step, the total number of individuals counted in the designated
time frame at the designated park and entrance is calculated. This adds
together pedestrians/skaters, bikers, those traveling in a vehicle
(calculated by multiplying number of vehicles by the given multiplier),
those traveling by bus (calculated by multiplying number of buses by the
given multiplier), those traveling by horseback, and those traveling by
boat. At this point, only scaling up the bus and vehicle counts has
occurred.

Expansion factors are assigned based on the park, park entrance class,
and the day type (weekend/holiday or weekday). The expansion factors are
\textbf{the number of trail entrance/timeslot combinations within a
particular usage type (high, medium or low; added in step 3)} at a
particular park in that season. Note that usage type is fluid throughout
time - a park entrance designated as `high' at 8 AM may be designated as
`low' at 6 PM. Therefore, the expansion factor is defined by a
combination of \textbf{BOTH} entrance \textbf{AND} time. The equation

\[ Trail\:entrances\:X\:Timeslots = Expansion\:Factor\:for\:Usage\:Classification\:at\:Park \]
describes this temporal-spatial scalar. These expansion factors are used
to scale counts on one trail/timeslot to all trail/timeslots of that
class within that park. Take, for example, Carver's Lake Waconia Park's
entrance \#1, which is medium usage from 8 to noon (two timeslots), and
high usage from noon to 8 PM (four timeslots). Because those four
timeslots at that entrance are the only time-place combinations at the
Lake Waconia Park designated as `high' usage, a count at a `high' usage
site at the Lake Waconia Park would be expanded by 1 trail X 4 timeslots
= \textbf{4} (expansion factor). However, the Dakota Rail Regional Trail
has \textbf{5 entrances} at which \textbf{all 6 timeslots} (8-10,
10-noon, noon-2, 2-4, 4-6, 6-8) are designated as `high' - so a Dakota
Rail Regional Trail count at any of these sites would be expanded by
\textbf{5 trails X 6 timeslots = 30 (expansion factor)}. If trails have
timeslots that differ in classification, trail entrances are multiplied
and then added. Minneapolis' Above the Falls Park, for example, has 5
entrances that are, at some point throughout the day, designated as
`high' usage. Entrances 1, 10 and 11 each have 2 high usage timeslots, 7
has 1, and entrance 8 has 6 high usage timeslots. This results in the
following equation

\[ (3*2)+(1*1)+(1*6) = 13 \] giving these Above the Falls entrances and
timeslots an expansion factor each of 13.

As of EOY 2018, the park entrance classification sampling matrix was
kept in a workbook entitled `XXXX Sample Class Workbook', where XXXX is
the year. Consult this workbook to determine the classifications of
trail entrances/times for the given year (usually the expansion factor
has already been calculated in this workbook as well). In 2019, the
process used to capture entrance usage designation was changed slightly
- classifications are now contained in a `tidy' format csv which eases
usage both directly in Excel (using filters and Pivot Tables) and in
R/RStudio (i.e.~sans conversion).

\subsection{Step 6: Compute expanded total of visitors at entrance and
by date and time of
day}\label{step-6-compute-expanded-total-of-visitors-at-entrance-and-by-date-and-time-of-day}

In this step, counts at a particular trail are scaled up to the park and
all-day level (via the expansion factors added in step 6) and then
further, up to the seasonal level (via the number of days in that
particular day type, be it weekend/holiday or weekday). The result is a
number that represents the seasonal count for that class of trail
(i.e.~entrance/time combo) and park, based on a single count that has
been successively scaled up. Note that adding \textbf{all} the expanded
totals in the dataframe now would be improper, as each trail entrance
has been scaled up to all trail entrances of that class in the given
park. Instead, these expanded totals will be averaged across groupings
and subsequently summed in step 9.

\subsection{Step 7: Add previous four years (or less) of data to current
year}\label{step-7-add-previous-four-years-or-less-of-data-to-current-year}

Because conducting counts of a representative nature is cost-prohibitive
(see the Met Council's July 2018 Report: METHODOLOGY FOR PRODUCING THE
REGIONAL PARKS SYSTEM ANNUAL USE ESTIMATE for more detailed
information), the newly processed year's worth of data is added to the
previous four year (or less) of available data to obtain a large enough
sample (i.e.~representative) over which to average (in the next step).
The methodology report notes that

\begin{quote}
The four-year approach reduces the effect of extreme weather or other
anomalies on each park's estimated visitation. It reduces the effects of
a temporary closure of a popular facility and road construction to the
facility. The four-year average also somewhat underestimates increased
visitation associated with a major new facility such as a swim pond or
play structure. By the fourth year of the facility's existence, its
visitation is fully represented in the data. The annual estimate makes
no attempt to correct for temporary closures or underestimates for
facility additions. All agencies in the regional system are adding
desirable facilities and experiencing temporary closures of facilities
that are being redeveloped. Therefore, it is believed that the four-year
average has negligible effects when comparing visitation across
agencies.
\end{quote}

\subsection{Step 8: Compute averages by Park, Trail Entrance/Timeslot
Classification, and Day type (Weekend/holiday or
Weekday)}\label{step-8-compute-averages-by-park-trail-entrancetimeslot-classification-and-day-type-weekendholiday-or-weekday}

In this step, the scaled-up estimates of a seasonal count, each created
based on a different base sample count, but many at identical trail
entrance/time classifications on identical day types, are averaged to
give \emph{one} unique summer count for each park or trail,
representative at the implementing agency level.


\end{document}
